%%%%%%%%%%%%%%%%%%%%%%%%%%%%%%%%%%%%%%%%%%%%%%%%%%%%%%%%%%%%%%%%%%%%%%%%%%%%%%%%%%%%%%%%%%%
%                                    PhDthesis v0.1                                       %
%                            By Quan Zou <quan.zou@gmail.com>                             %
%                            Version 0.1 released 05/01/2014                              %
%%%%%%%%%%%%%%%%%%%%%%%%%%%%%%%%%%%%%%%%%%%%%%%%%%%%%%%%%%%%%%%%%%%%%%%%%%%%%%%%%%%%%%%%%%%
%
%
%------------------------------------------------------------------------------------------
% thesis chapter
%------------------------------------------------------------------------------------------
\chapter{Model of Rock}
\hskip 0.5in

\section{Model 1}

Here is the equation~\ref{model:equ00} for {\bf RBL}, {\ie} {\rbl}~\cite{Smith2010}. 
\begin{equation}
 s = \sqrt{\frac{ \sum (x_{i} - \bar{x})^2}{n - 1}}  \label{model:equ00} ~ .  
\end{equation}

which implies that table~\ref{model:table00}.

\begin{table}[h!]
\centering
 \begin{tabular}{||c c c c||} 
 \hline
 Rock & Col1 & Col2 & Col3 \\ [0.5ex] 
 \hline\hline
 1 & 6 & 87837 & 787 \\ 
 2 & 7 & 78 & 5415 \\
 3 & 545 & 778 & 7507 \\
 4 & 545 & 18744 & 7560 \\
 5 & 88 & 788 & 6344 \\ [1ex] 
 \hline
 \end{tabular}
\caption{Rock table}
\label{model:table00}
\end{table}

